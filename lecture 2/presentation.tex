\documentclass[a4paper, 11pt]{beamer}

\usepackage{polski}
\usepackage[utf8]{inputenc}

\mode<presentation> {
	\usetheme{Frankfurt}
	\setbeamercovered{transparent}
	\usecolortheme{default}
}

\title{Ekonometria Finansowa}
\subtitle{Jednowymiarowe modele szeregów czasowych}
\author{mgr Paweł Jamer\thanks{pawel.jamer@gmail.com}}

\begin{document}

	\begin{frame}
		\titlepage
	\end{frame}
	
	\section{Biały szum}
	
	\begin{frame}{Biały szum}
		\begin{block}{\textbf{Biały szum}}
			Białym szumem nazwiemy szereg czasowy $\epsilon_t$ niezależnych zmiennych losowych o tym samym rozkładzie taki, że \begin{eqnarray*}
				\mathbb{E}\left(\epsilon_t\right) & = & 0,\\
				\mbox{Var}\left(\epsilon_t\right) & = & \sigma^2.
			\end{eqnarray*} Biały szum oznaczać będziemy symbolem $\mbox{WN}\left(0, \sigma^2\right)$.
		\end{block}
		\begin{alert}{\textbf{Uwaga}}
			Bardziej złożone modele szeregów czasowych wykorzystują biały szum do opisu niepewności pomiaru opisywanych przez nie wielkości.
		\end{alert}
	\end{frame}
	
	\section{Błądzenie losowe}
	
	\begin{frame}{Błądzenie losowe}
		\begin{block}{\textbf{Błądzenie losowe (bez dryftu)}}
			Szereg czasowy $p_t$ nazwiemy błądzeniem losowym bez dryftu, jeżeli spełnia on równanie \[
				p_t = p_{t-1} + \epsilon_t,
			\] gdzie
			\begin{itemize}
				\item $\epsilon_t$ --- biały szum.
			\end{itemize}
		\end{block}
		\begin{alert}{\textbf{Uwaga.}}
			Uzupełniając powyższy wzór o niezerową stałą $\alpha$ \[
				p_t = \alpha + p_{t-1} + \epsilon_t
			\] uzyskujemy proces błądzenia losowego z dryftem.
		\end{alert}
	\end{frame}
	
	\section{Modele ARMA}

	\begin{frame}{Proces MA}
		Zdefiniujmy operator \[
			\theta\left(B\right) = I + \theta_{1} B + \ldots +\theta_{q} B^{q},
		\] gdzie $q \in \mathbb{Z}_{+}.$
		\begin{block}{\textbf{Proces MA}}
			Słabo stacjonarny szereg czasowy $X_t$ nazwiemy procesem MA (średniej ruchomej) rzędu $q,$ jeżeli spełnia on równanie \[
				X_t = \theta\left(B\right) \epsilon_{t},
			\] gdzie $\epsilon_{t} \sim \mbox{WN}\left(0, \sigma^2\right).$ Proces MA rzędu $q$ oznaczać będziemy symbolem $\mbox{MA}\left(q\right)$.
		\end{block}
	\end{frame}
	
	\begin{frame}{Proces AR}
		Zdefiniujmy operator \[
			\varphi\left(B\right) = I - \varphi_{1} B - \ldots - \varphi_{p} B^{p},
		\] gdzie $p \in \mathbb{Z}_{+}.$
		\begin{block}{\textbf{Proces AR}}
			Słabo stacjonarny szereg czasowy $X_t$ nazwiemy procesem AR (autoregresyjnym) rzędu $p,$ jeżeli spełnia on równanie \[
				\varphi\left(B\right) X_t = \epsilon_{t},
			\] gdzie $\epsilon_{t} \sim \mbox{WN}\left(0, \sigma^2\right).$ Proces AR rzędu $p$ oznaczać będziemy symbolem $\mbox{AR}\left(p\right)$.
		\end{block}
	\end{frame}
	
	\begin{frame}{Proces ARMA}
		\begin{block}{\textbf{Proces ARMA}}
			Słabo stacjonarny szereg czasowy $X_t$ nazwiemy procesem $\mbox{ARMA}\left(p, q\right),$ jeżeli spełnia on równanie \[
				\varphi\left(B\right) X_t = \theta\left(B\right) \epsilon_{t},
			\] gdzie $\epsilon_{t} \sim \mbox{WN}\left(0, \sigma^2\right).$
		\end{block}
	\end{frame}
	
	\begin{frame}{Proces ARIMA}
		\begin{block}{\textbf{Proces ARIMA}}
			Szereg czasowy $X_t$ nazwiemy procesem $\mbox{ARIMA}\left(p, d, q\right),$ jeżeli
			szereg czasowy $\Delta^{d} X_t$ jest procesem $\mbox{ARMA}\left(p, q\right).$
		\end{block}
		Bezpośrednio z powyższej definicji wynika, że proces $\mbox{ARIMA}\left(p, d, q\right)$ charakteryzuje następujące równanie: \[
			\varphi\left(B\right)\left(\Delta^{d} X_{t}\right) = \theta\left(B\right) \epsilon_{t}.
		\]
	\end{frame}
	
	\begin{frame}{Proces SARIMA}
		\begin{block}{\textbf{Proces ARIMA}}
			Szereg czasowy $X_t$ nazwiemy procesem $\mbox{ARIMA}\left(p, d, q\right),$ jeżeli
			szereg czasowy $\Delta^{d} X_t$ jest procesem $\mbox{ARMA}\left(p, q\right).$
		\end{block}
		Bezpośrednio z powyższej definicji wynika, że proces $\mbox{ARIMA}\left(p, d, q\right)$ charakteryzuje następujące równanie: \[
			\varphi\left(B\right)\left(\Delta^{d} X_{t}\right) = \theta\left(B\right) \epsilon_{t}.
		\]
	\end{frame}
	
	\begin{frame}{Multiplikatywny proces ARMA}
		\begin{block}{\textbf{Proces ARIMA}}
			Szereg czasowy $X_t$ nazwiemy procesem $\mbox{ARIMA}\left(p, d, q\right),$ jeżeli
			szereg czasowy $\Delta^{d} X_t$ jest procesem $\mbox{ARMA}\left(p, q\right).$
		\end{block}
		Bezpośrednio z powyższej definicji wynika, że proces $\mbox{ARIMA}\left(p, d, q\right)$ charakteryzuje następujące równanie: \[
			\varphi\left(B\right)\left(\Delta^{d} X_{t}\right) = \theta\left(B\right) \epsilon_{t}.
		\]
	\end{frame}
	
	\section{Modele GARCH}
	
	\begin{frame}{Model ARCH}
	\end{frame}
	
	\begin{frame}{Model GARCH}
	\end{frame}
	
	\begin{frame}{Model GARCH-M}
	\end{frame}
	
	\begin{frame}{Model EGARCH}
	\end{frame}
	
	\begin{frame}{Model TGARCH}
	\end{frame}
	
	\section{Model ECM}
	
	\begin{frame}{Definicja}
		\begin{block}{\textbf{Model korekty błędem (ECM)}}
			\[
				\Delta y_{t} =
					\mu +
					\alpha \left(y_{t-1} - \beta_{0} - \beta_{1} x_{t-1}\right) + 
					\sum_{i=1}^{k-1} \theta_{i} \Delta y_{t-i} + 
					\sum_{i=0}^{k-1} \gamma_{i} \Delta x_{t-i} + 
					\epsilon_{t}
			\]
		\end{block}
		\textbf{Interpretacja:}
		\begin{itemize}
			\item $y_{t-1} = \beta_{0} + \beta_{1} x_{t-1}$ --- równanie równowagi
				długookresowej,
			\item $y_{t-1} - \beta_{0} - \beta_{1} x_{t-1}$ --- odchylenie od 
				równowagi długookr.,
			\item $\alpha$ --- współczynnik opisujący szybkość dostosowywania się 
				zmiennej objaśnianej do poziomu równowagi długookresowej (w 
				stabilnym modelu $\alpha < 0$).
			\item $\theta_{i}, \gamma_{i}$ --- współczynniki opisujące dynamikę
				krótkookresową.
		\end{itemize}
	\end{frame}
	
	\begin{frame}{Stosowalność}
		\begin{alert}{\textbf{Uwaga.}}
			Twierdzenie Grangera o reprezentacji gwarantuje nam możliwość
			zastosowania mechanizmu korekty błędem względem skointegrowanych
			szeregów czasowych.
		\end{alert}
	\end{frame}
	
	\begin{frame}{Estymacja}
		\begin{enumerate}
			\item Estymacja parametrów równania równowagi długookresowej \[
				y_{t-1} = \beta_{0} + \beta_{1} x_{t-1}.
			\]
			\item Skonstruowanie szeregów czasowych \begin{eqnarray*}
				\epsilon_{t} & = & y_{t} - \beta_{0} - \beta_{1} x_{t},\\
				\Delta x_{t} & = & x_{t} - x_{t-1},\\
				\Delta y_{t} & = & y_{t} - y_{t-1}.\\
			\end{eqnarray*}
			\item Estymacja parametrów równania modelu korekty błędem \[
				\Delta y_{t} =
					\mu +
					\alpha \epsilon_{t-1} + 
					\sum_{i=1}^{k-1} \theta_{i} \Delta y_{t-i} + 
					\sum_{i=0}^{k-1} \gamma_{i} \Delta x_{t-i} + 
					\epsilon_{t}
			\]
		\end{enumerate}
	\end{frame}

	\section*{}

	\begin{frame}
		\center
		\Huge \bfseries
		Pytania?
	\end{frame}

	\begin{frame}
		\center
		\Huge \bfseries
		Dziękuję za uwagę!
	\end{frame}

\end{document}