\documentclass[a4paper, 11pt]{beamer}

\usepackage{polski}
\usepackage[utf8]{inputenc}

\mode<presentation> {
	\usetheme{Frankfurt}
	\setbeamercovered{transparent}
	\usecolortheme{default}
}

\title{Ekonometria Finansowa}
\subtitle{Informacje Organizacyjne}
\author{mgr Paweł Jamer\thanks{pawel.jamer@gmail.com}}

\begin{document}

	\begin{frame}
		\titlepage
	\end{frame}

	\begin{frame}{Wymagania wstępne}
		\textbf{Wymagania wstępne:}
		\begin{itemize}
			\item podstawy programowania (r-language),
			\item rachunek prawdopodobieństwa,
			\item statystyka matematyczna,
			\item ekonometria dynamiczna.
		\end{itemize}
	\end{frame}
	
	\begin{frame}{Zakres materiału}
		\begin{enumerate}
			\item Zarys zagadnień związanych z rynkiem finansowym.
			\item Jednowymiarowe modele szeregów czasowych.
			\item Wielowymiarowe modele szeregów czasowych.
			\item Dekompozycja szeregów czasowych.
			\item Nieliniowe modele szeregów czasowych.
			\item Teoria efektywności informacyjnej rynku.
			\item Metody pomiaru ryzyka.
			\item Analiza portfelowa.
		\end{enumerate}
	\end{frame}
	
	\begin{frame}{Zaliczenie (punkty)}
		\begin{enumerate}
			\item \textbf{Teoria} (30 punktów):
			\begin{enumerate}
				\item \textbf{kartkówki} (15 punktów):
				\begin{itemize}
					\item na początku każdych zajęć,
					\item obowiązuje materiał z zajęć poprzednich,
				\end{itemize}
				\item \textbf{egzamin teoretyczny} (15 punktów):
				\begin{itemize}
					\item część egzaminu końcowego,
					\item pytania otwarte z całości teorii.
				\end{itemize}
			\end{enumerate}
			\item \textbf{Praktyka} (70 punktów):
			\begin{enumerate}
				\item \textbf{projekty} (30 punktów):
				\begin{itemize}
					\item dwa w ciągu semestru,
					\item przydzielone na drugich i szóstych zajęciach,
					\item czas na oddanie to około 30 dni,
				\end{itemize}
				\item \textbf{egzamin praktyczny} (40 punktów):
				\begin{itemize}
					\item część egzaminu końcowego,
					\item zadania rozwiązywane na komputerach.
				\end{itemize}
			\end{enumerate}
			\item \textbf{Szeroko pojęta aktywność}:
			\begin{enumerate}
				\item zadania dodatkowe, projekty, prezentacje,
				\item inwencja własna.
			\end{enumerate}
		\end{enumerate}
	\end{frame}
	
	\begin{frame}{Zaliczenie (ocena końcowa)}
		\textbf{Ocena końcowa}:
		\begin{itemize}
			\item \textbf{5.0} $\rightarrow$ od \textbf{91} punktów,
			\item \textbf{4.5} $\rightarrow$ od \textbf{81} do \textbf{90} punktów,
			\item \textbf{4.0} $\rightarrow$ od \textbf{71} do \textbf{80} punktów,
			\item \textbf{3.5} $\rightarrow$ od \textbf{61} do \textbf{70} punktów,
			\item \textbf{3.0} $\rightarrow$ od \textbf{51} do \textbf{60} punktów.
		\end{itemize}
		\textbf{Warunki dodatkowe}:
		\begin{itemize}
			\item minimum 51 punktów z części formalnej (teoria + praktyka).
		\end{itemize}
	\end{frame}
	
	\begin{frame}{Literatura}
		\begin{enumerate}
			\item Banaszczak-Soroka U., Dybał M., Homa M., Jakubowski S., Zawadzka P.; Rynki finansowe: organizacja, instytucje, uczestnicy; C. H. Beck, Warszawa 2014.
			\item Jajuga K., Jajuga T.; Inwestycje: instrumenty finansowe, aktywa niefinansowe, ryzyko finansowe, inżynieria finansowa; PWN, Warszawa 2014.
			\item Osińska M.; Ekonometria Finansowa; PWE, Warszawa 2006.
			\item Starzeński O.; Analiza rynków finansowych; C. H. Beck, Warszawa 2011.
			\item ...
		\end{enumerate}
	\end{frame}
	
	\begin{frame}{Materiały}
		\Huge\bfseries
		\centering
		http://jamer.pl
	\end{frame}

\end{document}